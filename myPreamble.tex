% Queen's Thesis Format
% (Borrowed from Dean Jin's BigDis.tex file, then heavily modified :)

% Michelle L. Crane, Queen's University, 2003

%*************************************************************************************************************
% DOCUMENT STYLE
%*************************************************************************************************************
\documentclass[12pt]{report}
%-------------------------------------------------------------------------------------------------------------
\usepackage{quthesis}        % the Queen's University dissertation style file
                             % Note:  In my thesis, I had many Listings, so I
                             % tweaked the old quthesis sty file to create a
                             % List of Listings in the table of contents.
                             % However, this version of quthesis does *not*
                             % include these modifications.

%I don't even use the fancyheadings - it looks nice enough without it
%\usepackage{fancyheadings}  % doesn't seem to change the headings at all!
%*************************************************************************************************************


%*************************************************************************************************************
% SPACING
%*************************************************************************************************************
\usepackage{setspace}        % for use of \singlespacing and \doublespacing
%*************************************************************************************************************


%*************************************************************************************************************
% VERBATIM
%*************************************************************************************************************
\usepackage{moreverb}        % Using this package to get better control of the
                             % verbatim environment, mostly for the use of the
                             % listing environment which puts line number
                             % beside each line.  Note that there has to be a number
                             % in each set of brackets, i.e., \begin{listing}[1]{1}.
                             % PDF info file is "The moreverb package" by
                             % Robin Fairbairns (rf@cl.cam.ac.uk) after
                             % Angus Duggan, Rainer Schopf and Victor Eijkhout, 2000/06/29.
%-------------------------------------------------------------------------------------------------------------
%\usepackage{verbatim}        % allows the use of \begin{comment} and \end{comment}
                             % as well as \verbatiminput{file}
                             % Note:  when using verbatim to input from a text file,
                             % such as a specification or code, use \begin{singlespacing}
                             % and \end{singlespacing}.  Also, tabs are not read
                             % properly, so the input file must only use spaces.

%                             \begin{comment}
%                             Can also use the verbatim package for
%                             comments like this...
%                             \end{comment}
%*************************************************************************************************************


%*************************************************************************************************************
% GLOSSARY
% Using a glossary is more than beginners need to know; leaving the packages, etc. here for now.
%*************************************************************************************************************
\usepackage[refpages]{gloss}  % for my glossary
                              % refpages shows the first page where the term occurs

%-------------------------------------------------------------------------------------------------------------
% Tell Latex to make a glossary
\makegloss

% These commands clean up the glossary - make the headings nice, and
% make the names stay in the right font.

% This command changes the way that the page reference number is
% shown.  In this case "See Page..." in brackets.
\renewcommand{\glosspage}[1]{ \emph{Page~#1.}}

% This command sets the glossary label to be the "word" in the
% glossary definition.  The #2 stands for the word.  #3 would be
% the definition, and #1 is the short form (I think).  If I comment
% out this command, then the labels are in a different font.
%\setglosslabel{#2}

% This command sets the glossary label to be the word, in bold, followed
% by the short form in brackets, if it exists.  This is where I can
% change the font to something else if desired.
\setglosslabel{\bfseries#1\ifglossshort{ (#3)}{}}

% This would be where I make changes to how the headings go.  For
% example, here the heading can be centered.
%\renewcommand{\glossheading}[1]{%
%    \stopglosslist %
%    \vspace{1pc}%
%    {\large\centering\bfseries#1\par}}

% This command will print the contents of the glossary, without
% headings between each letter.
\renewcommand{\glossheading}[1]{}

% Make changes to the environment, but I don't know exactly
% what it does...
%\renewenvironment{glosslist}
%    {\begin{description}}
%    {\end{description}}
%*************************************************************************************************************


%*************************************************************************************************************
% INDEX
% Also possible to make an index; didn't use for my thesis.
%*************************************************************************************************************
\usepackage{makeidx}         % to make the index
%-------------------------------------------------------------------------------------------------------------
% Tell Latex to make an index
\makeindex
%*************************************************************************************************************


%*************************************************************************************************************
% MATH STUFF
%*************************************************************************************************************
\usepackage{amsmath,amssymb}         % to make nice equations
%-------------------------------------------------------------------------------------------------------------
\usepackage{amsthm}          % to make nice theorem, i.e., definition

% Using the amsthm package, define a new theorem environment for my
% definition.  * means don't number it.
\newtheorem*{definition}{Definition}
%-------------------------------------------------------------------------------------------------------------
\usepackage{cases}           % to make numbered cases (equations)
%-------------------------------------------------------------------------------------------------------------
\usepackage{calc}            % Used with the Ventry environment defined below.
%*************************************************************************************************************


%*************************************************************************************************************
% FLOATS AND FIGURES % declared later
%*************************************************************************************************************
%\usepackage{graphicx}        % for graphic images (use \includegraphics[...]{file.eps})
%-------------------------------------------------------------------------------------------------------------
\usepackage{tocloft}% for table of figures/tables/content
       \setlength{\cftfignumwidth}{3em}
%       \renewcommand{\cftchapfont}{Chapter }
%\usepackage{subfigure}% for subfigures (figures within figures)
%-------------------------------------------------------------------------------------------------------------
\usepackage{boxedminipage}   % to make boxed minipages, i.e., boxes around figures
%-------------------------------------------------------------------------------------------------------------
\usepackage{rotate}          % for use of \begin{sideways} and \end{sideways}
%-------------------------------------------------------------------------------------------------------------
\usepackage{float}           % Using this package to get better control of my floats
                             % including the ability to define new float types for
                             % my specification and code listings.
                             % DVI info file is "An Improved Environment for Floats"
                             % by Anselm Lingnau, lingnau@tm.informatik.uni=frankfurt.de
                             % 1995/03/29.

% Define new float styles here
% Ruled style for examples
%\floatstyle{ruled}
%\newfloat{Example}{h}{lop}[chapter]

% Style of float used for code listings
\usepackage{listings} % this line is added by Xiaodong for self-defined code listing
\floatstyle{ruled}
\newfloat{Listing}{H}{lis}[chapter]

                             % Note:  The listings don't have space between the chapters, unlike
                             % the standard list of tables etc.  At the end, copy the spacing
                             % commands from the .toc file and insert into the .lis file.  Then,
                             % DO NOT LATEX it again, simply go to the DVI viewer!
%*************************************************************************************************************
% TABLES
%*************************************************************************************************************
\usepackage{tabularx}        % Package used to make variable width-columns, i.e.,
                             % column widths are changed to fit the maximum width
                             % and text is wrapped nicely.

\usepackage{threeparttable}
%*************************************************************************************************************
% CAPTIONS
%*************************************************************************************************************
\usepackage[hang]{caption}   % Package used to make my captions 'hang', i.e., wrap
                             % around, but not under the name of the caption.
%-------------------------------------------------------------------------------------------------------------
% Find that the captions are too far from my verbatim figures, but if
% I change it to 0, then the captions are too close for my other types
% of figures.  Maybe set each one separately?
%\setlength{\abovecaptionskip}{1ex}

%\setlength{\textfloatsep}{1ex plus1pt minus1pt}

%\setlength{\intextsep}{1ex plus1pt minus1pt}

%\setlength{\floatsep}{1ex plus1pt minus1pt}
%*************************************************************************************************************


%*************************************************************************************************************
% MISCELLANEOUS
%*************************************************************************************************************
\usepackage{layout}          % useful for determining the margins of a document
                             % use with \layout command
%-------------------------------------------------------------------------------------------------------------
\usepackage{changebar}       % Way of indicating modifications by putting bars in the
                             % margin.  Read about it in "The Latex Companion".
%*************************************************************************************************************


%*************************************************************************************************************
% REFERENCES ETC.
%*************************************************************************************************************
\usepackage{varioref}        % Better page references, e.g., "on preceding page", etc.
                             % \vref{key} Create an enhanced reference.
                             % \vpageref[text]{key} Create an enhanced page reference.
                             % \vrefrange{key}{key} Create an enhanced range of references.
                             % \vpagerefrange[text]{key}{key} Create an enhanced range of page references.
                             % Note: doesn't really work for consecutive pages.

% Renewing the text for before and after, because I don't like the default flip-flopping one.
% And 'on the page before' sounds dumb!

\renewcommand{\reftextafter}{on the next page}
\renewcommand{\reftextbefore}{on the previous page}
%-------------------------------------------------------------------------------------------------------------
\usepackage{url}             % for use of \url - pretty web addresses
% %*************************************************************************************************************
% % HYPERLINKS (must be last)
% %*************************************************************************************************************
% \usepackage[]{hyperref}
% \usepackage[dvips,bookmarks]{hyperref}
%                              % Neat package to turn href, ref, cite, gloss entries
%                              % into hyperlinks in the dvi file.
%                              % Make sure this is the last package loaded.
%                              % Use with dvips option to get hyperlinks to work in ps and pdf
%                              % files.  Unfortunately, then they don't work in the dvi file!
%                              % Use without the dvips option to get the links to work in the dvi file.
%
%                              % Note:  \floatstyle{ruled} don't work properly; so change to plain.
%                              % Not as pretty, but functional...
%                              % The bookmarks option sets up proper bookmarks in the pdf file :)
%
% % Need this command to allow hyperref to play nicely with gloss; otherwise
% % almost every \gloss will cause an error...
% \renewcommand{\glosslinkborder}{0 0 0}
% %*************************************************************************************************************


%*************************************************************************************************************
% MISCELLANEOUS COMMANDS AND ENVIRONMENTS
%*************************************************************************************************************
% Use this command to show more table of contents - used when playing
% with the draft outline
% I think it should be about 2???
\setcounter{tocdepth}{2}
%*************************************************************************************************************
% Environment definition I found in the "The Latex Companion".  Used to
% create a list environment where the indenting is the same for all of the
% entries, regardless of their length.  Note:  must \usepackage{calc}.
\newenvironment{Ventry}[1]%
    {\begin{list}{}{\renewcommand{\makelabel}[1]{\textbf{##1}\hfil}%
        \settowidth{\labelwidth}{\textbf{#1:}}%
        \setlength{\leftmargin}{\labelwidth+\labelsep}}}%
    {\end{list}}
%*************************************************************************************************************

%*************************************************************************************************************
% MY DEFINED COMMANDS
%*************************************************************************************************************
% Command that I can use to create notes in the margins;
% adapted from Juergen's META tag
\newcommand{\meta}[1]{\begin{singlespacing}
{\marginpar{\emph{\footnotesize Note: #1}}}\end{singlespacing}}
%*************************************************************************************************************
% Command that I can use to create lined headings
\newcommand{\heading}[1]{\bigskip \hrule \smallskip \noindent \texttt{#1} \smallskip \hrule}
%*************************************************************************************************************
% Command that I can use for reading in a file, verbatim, with line
% numbers printed along the left side.  The parameter is the file name.
\newcommand{\fileinnum}[1]{
    \begin{singlespacing} {\footnotesize
    \begin{listinginput}[1]{1}{#1}\end{listinginput}
    }\end{singlespacing}
}
%*************************************************************************************************************
% Command that I can use for reading in a file, verbatim, with NO line
% numbers, but in a smaller font.  The parameter is the file name.
\newcommand{\filein}[1]{
    \begin{singlespacing}{\footnotesize
    \begin{verbatiminput}{#1}\end{verbatiminput}
    }\end{singlespacing}
}
%*************************************************************************************************************
% Command that I can use for reading in a file, verbatim, with NO line
% numbers, but in a smaller font.  The parameter is the file name.
\newcommand{\fileinsmall}[1]{
    \begin{singlespacing}{\scriptsize
    \begin{verbatiminput}{#1}\end{verbatiminput}
    }\end{singlespacing}
}
%*************************************************************************************************************
% Dean't 'notesbox' command.  Needs setspace package.
%   Usage: \notesbox{This is a note.}
%
\newcommand{\notesbox}[1]{
%     \ \\
      \singlespacing
      \noindent\begin{boxedminipage}[h]{\textwidth}{\sf{#1}}\end{boxedminipage}
      \doublespacing
}


%**********************************
% My own definitions for shorthand
%**********************************
\usepackage{xcolor}
%\usepackage{amsmath}
\usepackage{bm}
%\usepackage{listings}
% % \textwidth 16cm \textheight 23.5cm
% \renewcommand{\baselinestretch}{1.2}
%\usepackage{graphicx}
\usepackage{cancel}
%\usepackage{subfigure}
\usepackage{psfrag}
\usepackage[greek,english]{babel} % added
\usepackage[utf8x]{inputenx} %
\usepackage[LGR, T1]{fontenc}
\usepackage{textcomp}  % defines \textmu, which is now what inputenx seems to use for ?? - probably due inpmath.. also \textdegree... but not \textrho
\usepackage{nomencl} % use nomenclatures
\renewcommand{\nomname}{List of Abbreviations}
\makenomenclature
\usepackage{CJK} % to input Chinese
%\AtBeginDvi{\input{zhwinfonts}}
\allowdisplaybreaks[3]% page breaks are allowed, more relaxed. go with amsmath package.
\usepackage{pdfsync}
%\usepackage[square, comma, sort&compress]{natbib}
\usepackage{cite}

%\usepackage{graphics}
%\usepackage{epsfig}
%\usepackage{color}
%\usepackage{multirow}
%\usepackage[colorlinks]{hyperref}
%\usepackage{fancyhdr}
%\usepackage{calc}
%\usepackage[numbers]{natbib}
%\usepackage{bibentry}
%\pagestyle{fancy}
%\headheight = 12pt

%\linespread{1.0}
%\fancyhead[R]{\thepage}
%\fancyfoot{}
%\hoffset =-1 cm
%\textwidth 424 pt
%\renewcommand{\headrulewidth}{0.4pt}
%\headwidth 424 pt
%\parindent 1 cm
% \usepackage[numbers]{natbib}

%% a try on fixing subfigure sytle regarding lofdepth command
 \makeatletter
 \let\c@lofdepth\relax
 \let\c@lotdepth\relax
 \makeatother
 \usepackage{subfigure}
% \usepackage[subfigure]{tocloft}
%\makeatletter
%\providecommand{\IfElsePackageLoaded}[3]{\@ifpackageloaded{#1}{#2}{#3}}
%\makeatother
%\IfElsePackageLoaded{subfig}
%	% IF subfig
%	{\usepackage[subfigure]{tocloft}}{	
%	% ELSE
%	\IfElsePackageLoaded{subfigure}
%		% IF subfigure
%		{\usepackage[subfigure]{tocloft}}
%	   % Else (No subfig nor subfigure)
%		{\usepackage{tocloft}}
%	}
%
%%%%%%%%% use pdf and eps at the same time %%%%%%
%\newif\ifpdf
%\ifx\pdfoutput\undefined
%   \pdffalse
%\else
%   \pdfoutput=1
%   \pdftrue
%\fi
\usepackage{ifpdf}

\ifpdf
   \usepackage{graphicx}
   \usepackage{epstopdf}
   \epstopdfsetup{suffix=}
   \DeclareGraphicsRule{.eps}{pdf}{.pdf}{`epstopdf #1}
   \pdfcompresslevel=9
\else
   \usepackage{graphicx}
\fi
%\DeclareGraphicsRule{.pdf}{eps}{}{`convert #1 eps:-}
%% �������� pdftex ���� latex �ĺ� \ifpdf
%%%%%%%%%%%%%%%%%%%%%%%%%%%%%%%%%%%%%%%%%%%%
%
%\ifpdf
%\DeclareGraphicsExtensions{.pdf,.png,.jpg,.mps}
%\else
%\DeclareGraphicsExtensions{.eps}
%\fi

%\newif\ifpdf
%\ifx\pdfoutput\undefined
%\pdffalse
%\else
%\pdfoutput=1
%\pdftrue
%\fi
%\ifpdf
%    \usepackage[pdftex]{graphicx}
%    \pdfcompresslevel=9
%\else
%    \usepackage{graphicx}
%    \DeclareGraphicsRule{.jpg}{eps}{.bb}{}
%    \DeclareGraphicsRule{.png}{eps}{.bb}{}
%\fi
%\ifx\pdfoutput
%\undefined \DeclareGraphicsRule{*}{eps}{*}{}
%\graphicspath{{Figs/}}
%\else
%\DeclareGraphicsRule{*}{pdf}{*}{}
%\graphicspath{{PDFfigs/}}
%\fi

% self-defined commands and definitions
% braket.sty          Macros for Dirac bra-ket <|> notation and sets {|}
%
\def\bra#1{\mathinner{\langle{#1}|}}
\def\ket#1{\mathinner{|{#1}\rangle}}
\def\braket#1{\mathinner{\langle{#1}\rangle}}
\def\Bra#1{\left<#1\right|}
\def\Ket#1{\left|#1\right>}
{\catcode`\|=\active
  \gdef\Braket#1{\left<\mathcode`\|"8000\let|\BraVert {#1}\right>}}
\def\BraVert{\egroup\,\mid@vertical\,\bgroup}
{\catcode`\|=\active
  \gdef\set#1{\mathinner{\lbrace\,{\mathcode`\|"8000\let|\midvert #1}\,\rbrace}}
  \gdef\Set#1{\left\{\:{\mathcode`\|"8000\let|\SetVert #1}\:\right\}}}
\def\midvert{\egroup\mid\bgroup}
\def\SetVert{\egroup\;\mid@vertical\;\bgroup}
% Some stuff deleted
% Macros for Dirac bra-ket <|> notation
\def\bra#1{\mathinner{\langle{#1}|}}
\def\ket#1{\mathinner{|{#1}\rangle}}
\def\braket#1{\mathinner{\langle{#1}\rangle}}
\def\ave#1{\mathinner{\langle{#1}\rangle}}
%
% END  braket.sty     Macros for Dirac bra-ket <|> notation and sets {|}

\newcommand{\greek}[1]{{\selectlanguage{greek}#1}} % will look for grmn font: tlmgr install cbfonts (65 MB)
\newcommand\lvec[1]{\accentset{\leftarrow}{#1}}
\newcommand{\dt}{\frac{d}{dt}}
\newcommand{\dtau}{\frac{d}{d\tau}}
\newcommand{\ssp}{\braket{\sigma^{+}(t)\sigma^{-}(t)}}
\newcommand{\aap}{\braket{a^{\dagger}(t)a(t)}}
\newcommand{\as}{\braket{a^{\dagger}(t)\sigma^{-}(t)}}
\newcommand{\sa}{\braket{a(t)\sigma^{+}(t)}}
\newcommand{\Hssp}{\braket{\sigma^{+}\sigma^{-}}}
\newcommand{\Haap}{\braket{a^{\dagger}a}}
\newcommand{\Has}{\braket{a^{\dagger}\sigma^{-}}}
\newcommand{\Hsa}{\braket{a\sigma^{+}}}
\newcommand{\adag}{a^{\dagger}}
\newcommand{\sigm}{\sigma^{-}}
\newcommand{\sigp}{\sigma^{+}}
\newcommand{\sigz}{\sigma^{z}}
\newcommand{\gp}{\gamma^{\prime}}
\newcommand{\oal}{\omega_a-\omega_0}
\newcommand{\ocl}{\omega_c-\omega_0}
\def\GFT{\overline{\bf G}}
\def\IT{\overline{\bf I}}
\def\TT{\overline{\bf T}}
\def\MT{\overline{\bf M}}
\def\AT{\overline{\bf A}}
\def\BT{\overline{\bf B}}
\def\fT{\overline{\bf f}}
\def\LT{\overline{\bf L}}
\def\alphaT{\overline{\bf \alpha}}
\def\GFTr{\overline{\bf G}\left(\mathbf{r},\mathbf{r}'\right)}
\def\GFTrw{\overline{\bf G}\left(\mathbf{r},\mathbf{r}';\omega\right)}
\def\rarg{\left(\mathbf{r}\right)}
\def\rargw{\left(\mathbf{r};\omega\right)}
\def\rrarg{\left(\mathbf{r},\mathbf{r}'\right)}
\def\rrargw{\left(\mathbf{r},\mathbf{r}';\omega\right)}
\def\rk{\left(\mathbf{r}_k\right)}
\def\rn{\left(\mathbf{r}_n\right)}
\def\rnrn{\left(\mathbf{r}_n,\mathbf{r}_n\right)}
\def\rnrk{\left(\mathbf{r}_n,\mathbf{r}_k\right)}
\def\rkrk{\left(\mathbf{r}_k,\mathbf{r}_k\right)}
\def\rkrn{\left(\mathbf{r}_k,\mathbf{r}_n\right)}
\def\br{\mathbf{r}}
\def\bG{\mathbf{G}}
\def\balpha{{\bm \alpha}}
\def\Erw{\hat{\mathbf{E}}^{(N)}(\mathbf{r},\omega)}
\def\E0{\hat{\mathbf{E}}^{(0)}(\mathbf{r},\omega)}
\def\Arw{\hat{\mathbf{A}}(\mathbf{r},\omega)}
\def\Enw0{\hat{\mathbf{E}}^{(0)}(\mathbf{r}_n,\omega)}
\def\Snw{\hat{\mathbf{S}}_n(\omega)}
\def\dnw{\hat{\mathbf{d}}_n(\omega)}
\def\Unw{\mathbf{U}_n(\omega)}
\def\unw{U_n(\omega)}
\def\Alphanw{{\bm \alpha}_n(\omega)}
\def\alphanw{\alpha_n(\omega)}
\def\Krrw{\mathbf{K}(\mathbf{r},\mathbf{r}',\omega)}
\def\Krrnw{\mathbf{K}(\mathbf{r},\mathbf{r}_n,\omega)}
\def\GTrrw{\mathbf{G}^T(\mathbf{r},\mathbf{r}',\omega)}
\def\Grrw{\mathbf{G}(\mathbf{r},\mathbf{r}',\omega)}
\def\Gn{\mathbf{G}^{(n)}}
\def\Gm1{\mathbf{G}^{(n-1)}}
\def\GN{\mathbf{G}^{(N)}}
\def\G0{\mathbf{G}^{(0)}}
\def\Gb{\mathbf{G}^0}
\def\G1{\mathbf{G}^{(1)}}
\def\Gi{\mathbf{G}^{(i)}}
\def\flamr{\mathbf{f}_\lambda(\mathbf{r})}
\def\rrn{\mathbf{r},\mathbf{r}_n}
\def\rnrn{\mathbf{r}_n,\mathbf{r}_n}
\def\rr{\mathbf{r},\mathbf{r}'}
\def\en{\mathbf{e}_n}
\def\eye{\mathbf{I}}


% define \onlinecite which is equivalent to RevTex style
\def\onlinecite{\cite}

%*************************************************************************************************************
% HYPERLINKS (must be last)
%*************************************************************************************************************
%\usepackage[]{hyperref}
\usepackage[bookmarks]{hyperref}
                             % Neat package to turn href, ref, cite, gloss entries
                             % into hyperlinks in the dvi file.
                             % Make sure this is the last package loaded.
                             % Use with dvips option to get hyperlinks to work in ps and pdf
                             % files.  Unfortunately, then they don't work in the dvi file!
                             % Use without the dvips option to get the links to work in the dvi file.

                             % Note:  \floatstyle{ruled} don't work properly; so change to plain.
                             % Not as pretty, but functional...
                             % The bookmarks option sets up proper bookmarks in the pdf file :)

% Need this command to allow hyperref to play nicely with gloss; otherwise
% almost every \gloss will cause an error...
\renewcommand{\glosslinkborder}{0 0 0}
%*************************************************************************************************************
